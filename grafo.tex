\begin{lstlisting}

    import csv
    from tqdm import tqdm
    from collections import deque
    
    csv.field_size_limit(1000000000)
    
    class LeyGrupo():
        def __init__(self, movimiento, nombre):
            self.movimiento = movimiento
            self.nombre = nombre
        
        def generar_movimiento(self):
            ciclo_alpha = self.crear_ciclo()
            pos_a = self.crear_posicion_arista()
            ciclo_beta = self.crear_ciclo()
            pos_b = self.crear_posicion_verice()
            self.movimiento = [ciclo_alpha, pos_a, ciclo_beta, pos_b]
            self.nombre = input("Ingrese el nombre (los giros) del movimiento: ")
            if not self.comprobar_restricciones(self.movimiento):
                print("Error: No se cumplieron las restricciones.")
                return None
            return self.movimiento
        '''Esta funcion genera un movimiento a partir de dos ciclos y dos posiciones'''
        
        def crear_ciclo(self):
            #ciclo = []
            uno = int(input("¿A qué posición va la posición 1? "))
            dos = int(input("¿A qué posición va la posición 2? "))
            tres = int(input("¿A qué posición va la posición 3? "))
            cuatro = int(input("¿A qué posición va la posición 4? "))
            ciclo = {1: uno, 2: dos, 3: tres, 4: cuatro}
            return ciclo
        
        def crear_posicion_arista(self):
            pos = []
            for i in range(4):
                color = int(input("Ingrese 0 o 1: "))
                if color != 0 and color != 1:
                    print("Error: Ingrese 0 o 1")
                    return None
                pos.append(color)
                i += 1
            return pos
        '''Esta funcion indica si el color de la arista está en su lugar (0) o no (1)'''
    
        def crear_posicion_verice(self):
            pos = []
            for i in range(4):
                color = int(input("Ingrese 0, 1 o 2: "))
                if color != 0 and color != 1 and color != 2:
                    print("Error: Ingrese 0, 1, o 2")
                    break
                pos.append(color)
                i += 1
            return pos
        '''Esta funcion indica si el color del vertice está en su lugar (0), si está girado 1/3 a la izquierda (1) o 2/3 a la izquierda (2), mirándolo desde la esquina'''
        
        def guardar_en_csv(self, archivo_csv):
            """Guarda el movimiento actual en un archivo CSV"""
            if self.movimiento and self.nombre:
                with open(archivo_csv, mode='a', newline='') as file:
                    writer = csv.writer(file)
                    writer.writerow([self.nombre, self.movimiento])
                print(f"Movimiento {self.nombre} guardado en {archivo_csv}")
            else:
                print("No hay movimiento generado para guardar.")
    
    
        def __str__(self):
            return f"El movimiento {self.nombre} es: {self.movimiento}"
        
        def componer_movimientos(self, m1, m2):
            nuevo_mov = []
            nuevo_mov.append(self.componer_ciclos(m1[0], m2[0]))
            nuevo_mov.append(self.componer_posiciones_mod2(m1[1], m2[1], m2[0]))
            nuevo_mov.append(self.componer_ciclos(m1[2], m2[2]))
            nuevo_mov.append(self.componer_posiciones_mod3(m1[3], m2[3], m2[2]))
            return nuevo_mov
            
        def componer_ciclos(self, c1, c2):
            nuevo_ciclo = {}
            for i in range(len(c1)):
                valor = c1[i+1]
                valor2 = c2[valor]
                nuevo_ciclo[i+1] = valor2
            return nuevo_ciclo
        
        def calcular_ciclo_inverso(self, c2):
            ciclo_inverso = {j: i for i, j in c2.items()}  # Generar inversa
            return dict(sorted(ciclo_inverso.items()))  # Ordenarlo
        
        def componer_posiciones_mod2(self, p1, p2, c2):
            ciclo_inverso = self.calcular_ciclo_inverso(c2)
    
            pos = [0] * 4
            for k in range(4):
                x = ciclo_inverso[k+1]
                pos[k] = p1[x-1]
    
            nueva_pos = [(pos[i] + p2[i]) % 2 for i in range(4)]
            return nueva_pos
    
        def componer_posiciones_mod3(self, p1, p2, c2):
            ciclo_inverso = self.calcular_ciclo_inverso(c2)
    
            pos = [0] * 4
            for k in range(4):
                x = ciclo_inverso[k+1]
                pos[k] = p1[x-1]
    
            nueva_pos = [(pos[i] + p2[i]) % 3 for i in range(4)]
            return nueva_pos
        
        def comprobar_restricciones(self, m1):
            # primero comprobamos que el sumatorio de las posiciones de las aristas sea cero
            if sum(m1[1]) % 2 != 0:
                return False
            # luego comprobamos que el sumatorio de las posiciones de los vertices sea cero
            if sum(m1[3]) % 3 != 0:
                return False
            
        def comparar_movimientos(self, m1, m2):
            if m1 == m2:
                return True
            else:
                return False
    
    
    # creamos un csv con los movimientos base de datos (son 34) solo si no existe
    # Este codigo solo se ejecuta una vez, para crear el csv
    def crear_csv():
        while True:
            print("-------------------- GENERANDO MOVIMIENTO --------------------")
            mov = LeyGrupo([], "")
            mov.generar_movimiento()
            mov.guardar_en_csv("movimientos.csv")
            
            continuar = input("¿Desea continuar? (s/n): ").strip().lower()
            if continuar != "s":
                break
            
    
    # creamos una clase Nodo para almacenar los movimientos
    class Nodo():
        def __init__(self, numero, nombre, movimiento):
            self.numero = numero
            self.nombre = nombre
            self.movimiento = movimiento
            self.adyacentes = []
        
        def agregar_adyacente(self, nodo, posicion):
            # Asegurarse de que no se agregue el propio nodo ni duplicados. ponemos el nodo adyacente en la posicion del movimiento aplicado para llegar a él
            if nodo != self and nodo not in self.adyacentes:
                self.adyacentes.insert(posicion, nodo)
                
        def __str__(self):
            return f"El movimiento es: {self.movimiento}"    
    
    # creamos un grafo con los movimientos del csv
    class Grafo():
        def __init__(self):
            self.nodos = {} # diccionario {numero del nodo: nodo}
            self.ley = LeyGrupo([], "")
        
        def agregar_nodo(self, numero, nombre, movimiento):
            if numero not in self.nodos:
                self.nodos[numero] = Nodo(numero, nombre, movimiento)
            else:
                print("Ya existe un nodo con ese número.")
        
        def agregar_arista(self, nodo1, nodo2, posicion):
            # añadir una arista que vaya del nodo 1 al nodo 2 en la posicion del movimiento aplicado para llegar al nodo2
            # evitamos crear una arista con el mismo nodo
            if nodo1 != nodo2:
                if nodo1 in self.nodos and nodo2 in self.nodos:
                    self.nodos[nodo1].agregar_adyacente(self.nodos[nodo2], posicion)            
        
        def mostrar_grafo(self):
            for num, nodo in self.nodos.items():
                print(f"Nodo {num} ({nodo.movimiento}): {[n.numero for n in nodo.adyacentes]}")           
       
        def combinar_grafo(self, nodos_fuente, nodos_destino, grafo_destino):
            """
            Combina los movimientos de dos grupos de nodos y guarda los nuevos en dos grafos, uno con toda la informacion y otro solo con la nueva
            """
            lg = LeyGrupo([], "")
            iteraciones = 0
    
            print(f"Combinando {len(nodos_fuente)} con {len(nodos_destino)} en un grafo aparte")
            
            # ponemos los nodos destino en el grafo destino
            for nodo in nodos_destino:
                grafo_destino.agregar_nodo(nodo.numero, nodo.nombre, nodo.movimiento)
    
            with tqdm(total=len(nodos_fuente) * len(nodos_destino), desc="Combinando", unit="iter") as pbar:
                for nodo1 in nodos_fuente:
                    for nodo2 in nodos_destino:
                        nuevo_mov = lg.componer_movimientos(nodo1.movimiento, nodo2.movimiento)
    
                        # Verificar si ya existe en el grafo destino
                        nodo_existente = next(
                            (n for n in grafo_destino.nodos.values() if lg.comparar_movimientos(nuevo_mov, n.movimiento)), 
                            None
                        )
    
                        if nodo_existente is None:
                            # Si no existe, crear un nuevo nodo en el grafo destino
                            nuevo_num = max(grafo_destino.nodos.keys(), default=-1) + 1
                            grafo_destino.agregar_nodo(nuevo_num, f"Nodo {nuevo_num}", nuevo_mov)
                            
                            # Conectar con el nodo de origen (nodo2) en el grafo destino
                            grafo_destino.agregar_arista(nodo1.numero, nuevo_num, nodo2.numero)
                        else:
                            # Si ya existe, solo añadir aristas
                            grafo_destino.agregar_arista(nodo1.numero, nodo_existente.numero, nodo2.numero)
    
                        iteraciones += 1
                        pbar.update(1)
    
            print(f"{len(grafo_destino.nodos)} movimientos almacenados en el grafo auxiliar")
    
        
        def guardar_grafo_csv(self, archivo_csv):
            with open(archivo_csv, mode='w', newline='') as file:
                writer = csv.writer(file)
                writer.writerow(['Numero', 'Nombre', 'Movimiento', 'Adyacentes'])
                
                for nodo in self.nodos.values():
                    writer.writerow([nodo.numero, nodo.nombre, nodo.movimiento, [n.numero for n in nodo.adyacentes]])
                print(f"Grafo guardado en {archivo_csv}")
    
    
    def cargar_grafo_de_csv(archivo_csv):
        grafo = Grafo()
        conexiones = {} # guardamos las conexiones temporalmente para añadirlas después
        
        with open(archivo_csv, newline='', mode='r', encoding="utf-8") as file:
            reader = csv.reader(file)
            # Se salta la cabecera
            next(reader)
    
            # Crear los nodos
            for row in reader:
                numero = int(row[0])
                nombre = row[1]
                movimiento = eval(row[2])  # Convertimos la cadena a su estructura original 
                adyacentes = eval(row[3])  # Esto es una lista de números de los nodos adyacentes
                
                grafo.agregar_nodo(numero, nombre, movimiento)
                conexiones[numero] = adyacentes # Guardamos las conexiones para después
    
            # crear las aristas
            for numero, adyacentes in conexiones.items():
                for i, adyacente in enumerate(adyacentes):
                    grafo.agregar_arista(numero, adyacente, i)                
        
        return grafo
    
    
    def cargar_movimientos_iniciales(archivo_csv):
        '''
        Cargamos los movimientos iniciales del archivo csv y los devolvemos en un diccionario.
        '''
        movimientos = {}
        with open(archivo_csv, newline='', mode='r', encoding="utf-8") as file:
            reader = csv.reader(file)
            for i, row in enumerate(reader):
                movimientos[i] = row[0]
        return movimientos
    
    import sys, os
    
    def resource_path(relative_path):
        """ Devuelve la ruta absoluta dentro del bundle de PyInstaller
            o la de tu carpeta normal si estás en desarrollo. """
        try:
            base_path = sys._MEIPASS
        except AttributeError:
            base_path = os.path.abspath(".")
        return os.path.join(base_path, relative_path)
    
    # cargamos los movimientos iniciales
    movimientos_iniciales = cargar_movimientos_iniciales(resource_path("movimientos.csv"))
    # cargamos el grafo
    grafo= cargar_grafo_de_csv(resource_path("grafo_final.csv"))
    
    def buscar_nodo(movimiento):
        """
        Busca un nodo en el grafo que tenga el mismo movimiento que el proporcionado.
        """
        for nodo in grafo.nodos.values():
            if nodo.movimiento == movimiento:
                return nodo.numero
        return None
            
    # ------------- CODIGO PARA CREAR EL GRAFO FINAL------------------
    # Este código se ejecuta una vez para crear el grafo final
    #cargo los movimientos del csv
    '''grafo = Grafo()
    
    # cargamos los movimientos iniciales del archivo csv en forma de grafo
    with open("movimientos.csv",newline= "", mode='r', encoding="utf-8") as file:
        reader = csv.reader(file)
        for i, row in enumerate(reader):
            nombre = row[0]
            movimiento = eval(row[1])
            grafo.agregar_nodo(i, nombre, movimiento)
            
    
    grafo_combinado1 = Grafo()
    grafo_combinado1.combinar_grafo(grafo.nodos.values(), grafo.nodos.values(), grafo_combinado1)
    #grafo.mostrar_grafo()
    #grafo_combinado1.mostrar_grafo()
    grafo_combinado1.guardar_grafo_csv("grafo_combinado1.csv")
    grafo.guardar_grafo_csv("grafo.csv")
    
    
    #opero los nuevos nodos con los nodos del grafo primero
    grafo_combinado2 = Grafo()
    grafo_combinado2.combinar_grafo(grafo_combinado1.nodos.values(), grafo.nodos.values(), grafo_combinado2)
    grafo_combinado2.guardar_grafo_csv("grafo_combinado2.csv")
    #grafo_combinado2.mostrar_grafo()
    
    #grafo_combinado3 = cargar_grafo_de_csv("grafo_combinado3.csv")
    
    grafo_combinado3 = Grafo()
    grafo_combinado3.combinar_grafo( grafo_combinado2.nodos.values(), grafo.nodos.values(), grafo_combinado3)
    grafo_combinado3.guardar_grafo_csv("grafo_combinado3.csv")
    grafo_combinado3.mostrar_grafo()
    
    grafo_combinado4 = Grafo()
    grafo_combinado4.combinar_grafo(grafo_combinado3.nodos.values(), grafo.nodos.values(), grafo_combinado4)
    grafo_combinado4.guardar_grafo_csv("grafo_combinado4.csv")
    
    grafo_final = Grafo()
    grafo_final.combinar_grafo(grafo_combinado4.nodos.values(), grafo.nodos.values(), grafo_final)
    grafo_final.guardar_grafo_csv("grafo_final.csv")'''
    
    
    # Algoritmo de búsqueda en anchura para encontrar el camino más corto a la identidad
    
    def buscar_identidad(nodo_inicial):
        ''' 
        Para cada nodo del grafo:
        - Si tiene conexión con la identidad (nodo 51), se añade a la lista de movimientos
        - Si no tiene conexión, elige el nodo con el numero más pequeño
        - Devuelve la lista de movimientos hasta llegar a la identidad
        '''
        movimiento_identidad = 51
        cola = deque()
        visitados = set()
        
        secuencia_movimientos = [] # guardamos los movimientos
        historial = [] # micky herramienta que usaremos más adelante (guarda los nodos visitados)
        cola.append(nodo_inicial)
        
        while cola:
            nodo_actual = cola.popleft()
            
            # Si ya lo visitamos, lo saltamos
            if nodo_actual in visitados:
                continue
            visitados.add(nodo_actual)
            
            # Lista para guardar los nodos adyacentes válidos
            nodos_adyacentes = grafo.nodos[nodo_actual].adyacentes
            
            # Buscar si el nodo 51 está en los adyacentes
            if any(nodo.numero == movimiento_identidad for nodo in nodos_adyacentes):
                for indice, nodo in enumerate(nodos_adyacentes):
                    if nodo.numero == movimiento_identidad:
                        secuencia_movimientos.append(movimientos_iniciales.get(indice, f"Movimiento_{indice}"))
                        historial.append(movimiento_identidad)
                        return secuencia_movimientos, historial  # Terminamos porque llegamos a 51
            
            # Si no encontramos el 51, buscamos el nodo con el número más pequeño
            nodo_mas_pequeno = min(nodos_adyacentes, key=lambda nodo: nodo.numero, default=None)
            
            if nodo_mas_pequeno:
                # Obtenemos el índice del movimiento que lleva a este nodo
                for indice, nodo in enumerate(nodos_adyacentes):
                    if nodo.numero == nodo_mas_pequeno.numero:
                        secuencia_movimientos.append(movimientos_iniciales.get(indice, f"Movimiento_{indice}"))
                        historial.append(nodo_mas_pequeno.numero)
                        cola.append(nodo_mas_pequeno.numero)
                        break  # Solo agregamos el primer camino encontrado
        
        return secuencia_movimientos, historial   

\end{lstlisting}